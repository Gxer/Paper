%%==================================================
%% chapter01.tex for BIT Master Thesis
%% modified by yang yating
%% version: 0.1
%% last update: Dec 25th, 2016
%%==================================================
\chapter{绪论}
\label{chap:intro}
\section{本论文研究的目的和意义}

导航是指将运动载体从甲地引导到乙地的技术\upcite{Qinyongyuan}。为了实现规定时间内将被导航目标引导至目的地,导航系统需要很多导航信息来确定载体当前状态,如姿态、速度、位置等\upcite{Lizibing}。惯性导航系统(INS)能够全天候提供三维定位定向信息,凭借其凭借其自主性、隐蔽性和精确性的优势,在海、陆、空、天等多个领域获得广泛应用。惯导导航系统从结构上分为平台惯导系统(PINS)和捷联惯导系统(SINS)两种。其中捷联惯导系统移除了复杂的物理平台,建立数学平台,通过复杂的导航算法设计来获取导航信息。由于具有适合于高动态环境、成本低、可靠性好和性能价格比高等优点,SINS在自主性高的场合受到了广泛的应用\upcite{Titterton2004Strapdown}。


\begin{figure}[!htp]
	\centering
	\includegraphics[width=0.75\textwidth]{figures/IMU}
	\caption{惯性导航系统}\label{fig:IMU}
\end{figure}


虽然捷联惯导系统具有很多优点,但SINS受工作原理限制,长期稳定性较差,位置、速度和姿态误差会随着时间发散\upcite{刘建业2010导航系统理论与应用}。惯性器件存在系统安装误差以及器件误差,包括陀螺仪、加速度计的零偏误差以及随机漂移等误差源,会导致系统的导航精度出现明显的下降。常值误差源,如陀螺加表零偏等,会造成等幅振荡的振荡误差。而随机误差源,如陀螺加表随机漂移等,将会造成振幅随时间均方根增长的振荡误差。在进行导航解算时,系统误差随时间积累,如不对误差进行估计补偿,定位误差会出现较快速增长甚至发散,导致导航精度下降。

传统的解决方案是构建惯性/多信息组合导航系统,通过卡尔曼滤波方法对时变误差参数进行在线估计和补偿。对于车载运动平台,一般做法是引入外参考信息,如GPS定位信息,与INS进行组合导航来提高定位精度。但是,传统组合导航系统仍存在不足。首先,这种方案在估计精度等方面存在明显的局限性,不能满足系统的实际使用需求。其次,该方案的可靠性仍值得商榷,相关滤波参数设置决定着补偿效果的优劣,不准确的滤波参数甚至会导致导航误差发散。因此,需要探究新的模式,以实现高精度导航定位。实际应用中,受卡尔曼滤波器设计效果、惯性器件误差源以及GPS定位精度影响,GPS/INS组合导航精度一般只能到达米级。而对于大地测量、地图测绘、惯导勘测、移动成图等应用领域,对于导航精度的要求一般到达亚米级甚至厘米级。在这些领域,如何充分地有效地利用相关辅助导航定位技术实现高精度导航定位成为了急需解决的问题。

考虑到大地测量、地图测绘等上述相关领域对于导航实时性要求不高,可以利用后处理技术来提升导航精度。后处理通常指以惯性信息为主进行的组合导航运算离线处理,即存储导航系统任务过程中的惯性信息和其它辅助信息,如GPS信息、里程计信息等,在任务结束后进行事后处理\upcite{林翰2014惯性基多信息在线后处理技术研究}。在不占用额外的系统硬件资源条件下,数据后处理以损失实时性需求的代价,可以获得更高精度的导航结果。本文采用基于回溯导航思想的事后处理技术来提升导航精度。所谓的回溯,即在正常导航解算的过程中存储一部分或全部数据,并以这些数据再次解算以得到更好的导航精度或者估计出导航中的部分误差参数\upcite{李万里2013惯性}。与传统在线标校方式相比,利用回溯技术,将惯性导航系统中传感器采样数据当作一组时间序列,对该序列按照时间先后顺序进行正逆向多次反复处理,并与辅助导航结果进行正向滤波、后向平滑的数据融合,可以实现高精度定位结果。影响回溯导航精度的主要因素包括回溯过程系统模型的建立、噪声未辨识参数最优估计方法等等。在回溯过程中,如何构建逆向解算模型、逆向滤波算法?如何减小系统模型不准确对滤波过程的影响?未知环境下噪声统计特性难以实时获取,如何构建噪声自辨识算法,实现包含未建模动态误差的参数最优估计?本文将分别针对上述三个问题展开进一步的分析和研究。


\section{国内外研究现状及发展趋势}
%\label{sec:***} 可标注label

\subsection{SINS/GPS事后处理算法}
%\label{sec:features}

SINS/GPS事后处理算法常用于大地测量、地图测绘、惯导勘测、移动成图等领域,实现高精度定位定向效果。高精度定位定向系统一般称为POS(Position Orientation System)。为保证导航精度,POS一般采用精度较高的导航器件,如激光惯导、高精度双频GPS接收机等。在此硬件基础上,通过SINS/GPS事后处理算法来进一步提升精度。

这里要做修改

为保证导航精度,POS 中一般采用高精度双频 GPS接收机,并通过差分GPS与SINS进行组合导航。差分GPS分为伪距差分和载波相位差分两类 [15] 。伪距差分称作 DGPS,是目前应用最广的一种差分技术,普通商用差分 GPS 都采用这种技术;而测绘用的GPS一般采用载波相位差分技术(KAR),包括实时差分(RTK)和事后差分两类。DGPS 定位精度在亚米级,定位精度随着用户到基准站的距离增加而有所降低;RTK 要求流动站的 GPS 接收机能够快速解算载波相位的模糊度,可以达到厘米级的定位精度;KAR 属于事后处理,定位精度不低于RTK。实时差分要求用户接收机不但要接收 GPS 轨道卫星信息,还要接收基准站信息,差分数据的连续性受基准站位置布局的影响;而事后差分只需用户接收机和基准站分别接收并保存 GPS 原始测量信息,所以不受二者通讯链路的限制,抗干扰能力更强。
在IMU 精度和 GPS 定位精度一定的条件下,车载 POS 数据处理主要包含实时 SINS/GPS 组合导航和事后平滑处理两个主要部分。车载 POS 本质上就是 SINS/GPS 组合导航系统,实时处理算法与成熟的 SINS/GPS 组合导航算法类似,因此这也成为国内技术人员研究的切入点。
车载 POS中 SINS/GPS 实时组合导航的最佳方案是采用紧组合策略,即 GPS 接收机无需进行定位解算,而是直接采用测量伪距和载波相位观测值与 SINS 的导航结果构成量测值。POS LV 系统即是在紧组合策略的基础上构建了 In-Fusion 组合滤波结构 [15] ,其主要优势在于:①可以采用 SINS 的解算结果对单个卫星的测量信息进行故障检测, 在可见星数少于 5 颗的情况下仍能进行故障检测;②当可见星数少于 4 颗时, GPS 接收机不能进行定位解算,但紧组合策略仍能利用这些信息进行组合导航,从而改善定位精度;③使用 SINS 的解算结果辅助解算模糊度的初值,即 IARTK 技术,可以减少对整周模糊度初始化的时间,当失锁结束后,可以快速恢复模糊度的解算,提供抗干扰能力强 的 RTK 定位。
由于国内GPS研究现状的限制,使用 SINS/GPS紧组合策略要达到测绘级的精度要求比较困难,所以目前国产POS系统主要集中在松组合研究,即位置、速度组合方式。

POS数据的事后处理通常采用平滑算法,平滑算法又可分为固定点平滑,固定滞后平滑和固定区间平滑三类 [16] 。POS系统需要输出整个测绘工作期间的导航参数,所以适合采用固定区间平滑算法。
双向平滑算法(TFS)是 Fraser和 Potter [17] 提出的一种最优固定区间平滑算法。文献[18]指出,传统的 TFS算法仅适用于线性系统,而导航解算方程具有强非线性的,因而不能直接使用。一种解决途径是采用基于扩展卡尔曼滤波(EKF)的 TFS方法,但存在缺点是不能准确估计 INS误差状态。文献 [19] 提出了改进的EKF平滑方法并用于管道地理坐标的惯性测量,文献 [20]也在城市车辆的 SINS/GPS组合导航中使用了改进 EKF平滑方法。
1965年, Rauch, Tung和 Striebel提出了另一种固定区间平滑算法 [21] ,称作RTS平滑 (RTSS)算法。数学上可以证明, RTSS算法和 TFS算法 是完全等价的 [22] ,所以RTSS算法也是一种最优固定区间平滑算法,且算法的平滑效果和TFS算法完全相同。由于 RTSS算法简洁有效,所以一经提出便广受关注。文献 [23] 最早将RTSS算法应用于惯性导航系统,文献 [24]使用该算法对惯导系统的对准精度进行评估,文献[25]利用该算法对 SINS/GPS组合数据进行了事后平滑处理。迄今为止, RTSS算法仍是使用最为广泛的固定区间平滑方法,加拿大 NovAtel公司的 Inertial Explorer事后处理软件也使用了该方法 [26] 。

此外,时候处理中通常结合捷联惯导的逆解算过程。基于“逆解算”思想的高精度定位定向导航系统,重点之一在于逆解算公式的推导、证明与应用。孙进[12]提出一种基于逆向导航解算和数据融合的SINS传递对准方法。文中提出两种观点:1)可以使用一段惯性传感器数据完成捷联惯性导航系统(SINS)的初始对准; 2)通过在数据融合中加入逆向—正向SINS解算结果和外部参考数据,可以缩短传感器误差的估计时间。基于以上两种观点,旨在不改变卡尔曼滤波器的情况下,短时间内估计陀螺漂移的一种快速传递对准的方法。在一个参考数据更新周期内,储存了惯性传感器数据和参考数据,并且执行了逆向—正向捷联解算。与此同时,当相应的捷联解算结束后,执行数据融合算法。在上述的更新周期内,由于加入逆向—正向捷联解算,陀螺漂移估计操作增加了两次并且缩短了其估计时间。覃方君[13]提出一种基于正逆向与降噪的捷联惯导改进快速对准方法。针对捷联惯性导航系统(INS)的快速对准问题,基于双向过程和惯性传感器的降噪方法,提出了一种改进的对准方法。利用双向过程(前向和逆向)反复处理保存的惯性测量单元(IMU)的数据序列实现快速对准,推导了一种新的前向与逆向对准关系。为了减少角随机游走误差的影响,基于小波变 换的降噪方法抑制光纤陀螺(FOGs)和加速度计噪声,给出陀螺罗经回路的改进方法的整个流程,并在自研的光纤陀螺捷联惯导系统上进行测试。严恭敏[14]提出一种逆向导航算法及其在捷联罗经动基座初始对准中的应用。根据捷联惯导系统的导航更新算法,详细推导了逆向导航算法。提出了捷联罗经动基座初始对准方案,它包括三个阶段:方位角未知情况下水平对准、粗略方位自对准和罗经方位对准。在导航计算机存储容量足够大并且计算能力足够强的条件下,通过逆向导航算法将动基座初始对准与位置导航有机结合起来,先进行捷联罗经动基座初始对 准和传感器采样数据存储,再利用逆向航位推算算法和正向航位推算算法,同时实现了初始对准和位置导航。最后, 利用车载试验验证了所提方法的有效性。由以上文献可知,逆解算在惯性导航领域已有应用,但多集中于初始对准方面。

\subsection{卡尔曼滤波系统误差模型研究现状}
%\label{sec:requirements}
 组合导航多采用Kalman滤波来提高导航精度[2]。在Kalman滤波中,采用迭代策略不断利用观测信息去修正状态预报值,不但能够较好地抑制状态扰动异常和模型误差对导航解的影响,而且能够充分挖掘观测信息中含有的状态参数最优估值信息,进一步提高了导航解的精度。但反复迭代方法仍然没有解决系统误差模型不准确以及噪声统计特性未知对Kalman滤波产生影响的问题。在惯性导航系统中,由于系统误差模型不准确以及随机噪声的统计特性难以精确获知等问题,使得标准Kalman滤波算法失去最优性,严重降低估计精度,甚至引起滤波发散。
 捷联惯导误差模型在分析导航系统的误差传播特性以及利用卡尔曼滤波进行初始对准方面具有重要作用。为了更好地对系统误差建模,许多学者做了大量工作,提出各种模型。常用的惯性导航系统线性模型有两种[27-28]: 角误差模型和 角误差模型。 角误差描述的是平台坐标系(对于捷联惯性导航则为数学平台坐标系)与真实导航坐标系间的误差角, 角误差描述的是平台坐标系(对于捷联惯性导航则为计算平台坐标系)与计算导航坐标系间的误差角,因此这两种误差模型又称为角误差模型。Benson[29]通过引入 、  角误差模型,给出了这两种模型的关系。Friedland[30]使用四元数给出惯性坐标系下捷联惯导姿态和速度误差分析。Shibata[31]给出当地水平坐标系下捷联惯导姿态误差的加性四元数形式以及相应的等效误差角形式。


\subsection{自适应卡尔曼滤波发展近况}

1960年,R.E. Kalman首次提出了卡尔曼滤波理论,该滤波理论标志着现代滤波理论的建立[17]。卡尔曼滤波是一种基于状态空间模型的递推最优估计方法。针对噪声特性满足高斯噪声特性的系统,卡尔曼滤波可以获得系统状态变量的最小均方误差估计。此外,卡尔曼滤波理论首次将现代控制理论中的状态空间思想与最优滤波理论相结合[4]。相对于过去的滤波方法,卡尔曼滤波可以对时变系统、非平稳信号和多维信号进行滤波估计。由于当时计算机计算能力和工作原理限制,很多传统的滤波算法并不适合在计算机上进行数字实现,而由于卡尔曼滤波是一种递推滤波算法,单次的计算量也相对较小,并且滤波过程可以分步计算和并行计算,便于在计算机上实现。由于以上特性,卡尔曼滤波被广泛应用于诸如导航制导[18]、信号通信[19]以及故障检测[20]等各个领域。
传统的卡尔曼滤波的约束条件比较严格。适用卡尔曼滤波的系统必须满足如下三个约束:1、系统必须是线性系统,且系统模型参数必需精确已知;2、系统的噪声特性必须满足高斯噪声特性,且两者之间必需彼此独立互不相关;3、滤波之前必须获取噪声的先验统计特性。
针对系统非线性问题,可采用扩展卡尔曼滤波(Extended Kalman Filter, EKF)、Sigma点卡尔曼滤波(Sigma Point Kalman Filter, SPKF)以及容积卡尔曼滤波(Cubature Kalman Filter, CKF)等滤波算法。上述算法中,扩展卡尔曼滤波通过对被估计状态进行扩维,将非线性系统线性化[20,21]。EKF原理简单明了,设计方便,然而EKF滤波过程中需要计算雅各比矩阵,计算量大,且当雅各比矩阵不准确时容易造成滤波发散[22-24]。Sigma点卡尔曼滤波(SPKF)是一类基于高斯分布的近似非线性滤波方法[23-28]。SPKF类里主要包括中心差分卡尔曼滤波(Central Difference Kalman Filter, CDKF)、无迹卡尔曼滤波(Unscented Kalman Filter, UKF)、平方根无迹卡尔曼滤波(Square-Root Unscented Kalman Filter, SRUKF)、平方根中心差分卡尔曼滤波(Square-Root Central Difference Kalman Filter, SRCDKF)等。其中以UKF最为著名。相比于传统EKF,UKF在保证精度的同时减少了计算量,因此实时性较EKF更好。容积卡尔曼滤波(CKF)最早由Simon Haykin于2009年提出[27],属于高斯贝叶斯滤波,因此CKF并不能在非高斯噪声系统中进行滤波估计。但是CKF因其具有实现简单、非线性逼近能力强等优点,与传统非线性算法比较仍具有较大优势。相较于 UKF,CKF数值精度更高;可采用平方根策略求解;滤波稳定性更优[27-29]。
针对噪声先验信息不足的情况,最为常用的算法是自适应滤波算法。
B. Widrow等人最早于70年代中期提出了自适应滤波算法,以解决在Q、R未知情况下卡尔曼滤波器无法保证其滤波稳定性以及估计精度严重下降的问题[30]。自适应滤波算法将自适应控制的思想引入滤波估计问题中。由于可以在线动态调整滤波模型参数,因此该方法对于系统的先验信息的依赖度较低,可以在系统噪声特性的先验信息不准甚至未知的情况下对系统状态进行估计。常用的自适应滤波算法有神经网络自适应滤波,模糊自适应滤波,极大后验估计(也称为Sage-Husa算法,1969)[30]、虚拟噪声补偿(shimura, 1978)[31]、动态偏差去耦估计(Firedland, 1969)[32]等。上述方法在传统卡尔曼滤波的基础上减少了对噪声先验信息的依赖程度,在一定程度上提高了卡尔曼滤波的鲁棒性。
Sage-Husa滤波算法(后简称为S-H算法)是一种带有噪声特性估计(系统量测噪声与过程噪声的期望和方差)的滤波算法。该方法通过在传统的卡尔曼滤波过程中加入噪声估计过程,通过新息序列对噪声特性进行估计,并在线修正滤波器参数,进而提高滤波精度。然而由于新息与上一时刻滤波结果有关,而当前时刻的滤波结果又会影响到下一时刻的噪声估计过程,进而影响下一时刻的状态估计与新息序列的计算精度。因此,该方法容易出现滤波结果误差较大,滤波过程发散,无法保证估计结果的Q、R矩阵的正定性等问题。
针对S-H算法中存在的问题,王忠在传统的S-H算法的基础上进行改进,通过在噪声特性估计过程中引入遗忘因子,借此来削弱过去时刻新息序列对噪声估计的影响,并且可以减弱新息序列中的误差项对滤波结果的影响,改善滤波器的动态性能[33]。在此基础上,考虑到新息序列可能对状态噪声和量测噪声特性的估计过程的影响有所不同,范科等学者将单一遗忘因子进行扩展,提出了多遗忘因子的S-H算法,通过增加遗忘因子个数,实现不同类型的噪声估计过程具有不同的特性,增强了算法的自适应性,提高了估计过程的收敛速度[34]。为了抑制S-H算法的滤波可能发散的问题,鲁平等学者将阈值限制加入S-H算法之中,通过阈值限制新息对估计过程的影响,进而抑制滤波发散程度,借此提高了S-H滤波算法的稳定性[35]。黄晓瑞等学者将S-H算法与强跟踪滤波进行结合,通过新息序列与已知信息的吻合程度作为性能指标,实现S-H算法与强跟踪滤波的切换,以此增强滤波算法的鲁棒性[36]。针对传统S-H算法两个协方差矩阵的非正定性问题,魏伟等学者提出了一种解决Q、R非正定的方法,并改进了S-H算法和遗忘因子的设计方法,通过去除量测噪声对过程噪声估计过程的影响,提高了估计精度,并且提高了收敛速度[37]。
除了S-H算法及其改进算法之外,有很多学者提出了新的自适应滤波算法。例如基于滤波残差的滑动窗口的量测噪声估计算法[38,39]。还有学者将大数定律用于噪声特性估计,提出基于大数定理的噪声估计算法[40-43]。针对新息序列中可能存在野值点,造成的滤波发散问题,许丽佳等学者提出了野值点的判断和剔除算法,并将其与自适应滤波进行结合,在某些特定环境中有较好效果[44]。此外,还有将模糊逻辑控制与自适应滤波结合,利用模糊控制律来选定自适应因子,抑制误差发散对于滤波器的影响,进而增强了滤波器的稳定性[45-51]。部分学者利用载体运动的特性,将运动约束加入状态方程中,通过增加先验信息对滤波的作用,进而优化滤波估计的结果[52]。也有将神经网络与自适应滤波进行结合的相关研究[53-55]。神经网络具有强大的非线性拟合能力,可映射任意复杂的非线性关系,此外还具有很强的鲁棒性。在网络训练良好的情况下,神经网络自适应滤波有着较好的表现。通过神经网络来修正状态估计,可以抑制异常扰动对滤波结果的影响。部分学者采用小波滤波与低通滤波结合,对信号进行预处理,增加新息序列的信噪比,该方法在一定程度上提高滤波估计的精度[56,57]。自校正滤波是基于ARMA新息模型的最优滤波算法,通过白噪声估计器实现对噪声特性的在线估计,进而提高噪声先验信息不准或不足情况下的滤波精度[58-60]。

\section{论文主要研究内容}
本文主要针对传统车载组合导航系统精度不足的问题,提出基于回溯导航的事后处理方案。针对回溯导航中系统模型不准确的问题,提出基于计算系下表示的改良系统误差模型,提高卡尔曼滤波系统模型精度;针对回溯导航中噪声先验统计特性未知问题,提出回溯过程中基于双向观测序列的噪声方差估计方法,解决噪声统计特性发生变化时滤波器估计精度下降的问题。最后设计整套回溯导航方案,并对其有效性进行了验证。本文整体结构如下:


第一章参考了研究相关的国内外文献,首先介绍了本文的研究背景,然后依次介绍了SINS/GPS事后处理算法的发展状况、卡尔曼滤波系统误差模型的研究现状以及自适应卡尔曼滤波发展近况。最后简述了本论文的整体结构。

第二章介绍了回溯导航的基本思想以及回溯导航算法方案。在回溯导航模式下进行了正逆向解算流程的推导,构建回溯解算模型。针对回溯导航模式下的双向平滑算法展开研究,对其可观测性进行了分析,并对平滑结果的无偏性进行证明通过仿真实验验证了该算法在回溯过程中的有效性。

第三章简要介绍了卡尔曼滤波器中传统系统误差模型的构建以及其存在的问题。针对系统模型不准确问题,提出了计算系下表示的改良系统误差模型,并分别给出改良系统误差模型中姿态、速度、位置误差传播方程的推导。最后通过仿真实验验证改良系统误差模型的有效性,并对滤波效果进行评估。

第四章介绍了基于噪声观测序列的系统过程噪声和量测噪声协方差矩阵的估计算法。通过系统增广模型扩维,给出了回溯过程中基于双向观测序列的噪声估计方差估计方法,并证明了估计算法的无偏性和一致性。设计了相应的自适应滤波算法,最后通过仿真实验验证滤波算法的有效性,并对滤波效果进行评估。

第五章给出回溯导航整体算法方案的设计,并搭建实际跑车平台采集跑车数据。通过实际跑车数据对改良系统误差模型、基于噪声观测序列的噪声估计方差估计方法进行验证。最后通过与传统卡尔曼滤波算法进行对比,评估了回溯导航整体算法方案的有效性。

论文最后的结论部分对前面几章的主要内容进行了总结和归纳,并总结出目前尚未完善的工作,为今后的进一步研究指明方向。

